% $Id: epigrafe.tex,v 1.1 2003/04/10 23:12:59 gweber Exp $

%%%%%%%%%%%%%%%%%%%%%%%%%%%%%%%%%%%%%
%% Epigrafe
%% Copyright 2003 Dehon Charles Regis Nogueira.
%% Este documento � distribu�do nos termos da licen�a
%% descrita no arquivo LICENCA que o acompanha.
%%%%%%%%%%%%%%%%%%%%%%%%%%%%%%%%%%%%%


%  Ep�grafe - � uma cita��o pertinente ao seu trabalho
%  ou que represente o seu modo de pensar. 
%  Resumindo, coloque uma frase que o(a) agrade.


\pretextualchapter{}

\vspace{17.5cm}
\begin{flushright}

\textit{``A atividade da engenharia, enquanto permanecer atividade, \\
	 pode levar a criatividade do homem a seu grau m�ximo; \\
	 mas, assim que o construtor p�ra de construir e se entrincheira \\
	 nas coisas que fez, as energias criativas se congelam, \\
	 e o pal�cio se transforma em tumba.'' \\ 
	\bfseries Marshall Berman}

\end{flushright}


