% $Id: agradecimentos.tex,v 1.1 2003/04/10 23:12:59 gweber Exp $

%%%%%%%%%%%%%%%%%%%%%%%%%%%%%%%%%%%%%
%% Agradecimentos
%% Copyright 2003 Dehon Charles Regis Nogueira.
%% Este documento � distribu�do nos termos da licen�a
%% descrita no arquivo LICENCA que o acompanha.
%%%%%%%%%%%%%%%%%%%%%%%%%%%%%%%%%%%%%

\chapter*{Agradecimentos}


% Agradecimentos - � so para as pessoas que contribuiram relevantemente
% para a elabora��o do trabalho

Dedico meus sinceros agradecimentos para:

-- o professor doutor Ant�nio S�rgio Bezerra Sombra, pela orienata��o e incentivo;

-- a equipe do Laborat�rio de �ptica N�o-Linear e Ci�ncia de Materiais do Departamento de F�sica da UFC, em especial aos colega doutorandos Rinaldo e Silva de Oliveira e Ana Fab�ola Leite Almeida, pela ajuda em diversos momentos;

-- o coordenador do curso de Mestrado em Engenharia e Ci�ncia de Materiais, professor doutor Lindberg Lima Gon�alves, pelo apoio sempre manifestado;

-- o professor doutor Hamilton Ferreira Gomes de Abreu, pela oportunidade de juntar-me � equipe da ANP e � professora doutora M�nica Cavalcante S� de Abreu pela ajuda, entre outras coisas, na obten��o de amostras;

-- o professor doutor Hosiberto Batista de Sant'Ana, pelo aux�lio nos t�picos referentes � Engenharia de Petr�leo;

-- a minha esposa J�lia, pela revis�o deste trabalho;

-- a Ag�ncia Nacional de Petr�leo, pela oportunidade de realiza��o deste trabalho;

-- o Departamento de Eletr�nica e Sistemas da UFPe, nas pessoas do professor doutor Edval J. P. Santos e do mestrando Victor Miranda da Silva, pela recep��o e aux�lio durante minha estada em Recife;

-- o Laborat�rio de Lubrificantes e Combust�veis da UFC, nas pessoas de St�lio Menezes Rola Jr. e Sandra Lidiane Mota da Silva, pelas amostras fornecidas e pelo aux�lio em diversos momentos de d�vida;

-- a Lubnor, na pessoa do engenheiro Paulo de Almeida Luz, pelas amostras de combust�vel fornecidas;

-- todos os colegas do Mestrado em Engenharia e Ci�ncia de Materiais da UFC.
