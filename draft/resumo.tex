\chapter{Resumo}

A linguagem de programa��o C++, apesar de ser auto-n�vel e possibilitar a produ��o de um c�digo de 
execu��o bastante eficiente, imp�e ao desenvolvedor um custo relativamente alto: o tempo necess�rio
para a compila��o do seu c�digo. T�cnicas agora dispon�veis, especialmente \textit{Template Metaprogramming},
enquanto promovem a gera��o de c�digo gen�rico, contribuem ainda mais para este custo.

Este trabalho tem como o objetivo o estudo e a cria��o de uma ferramenta que
distribua o processo de compila��o entre v�rias m�quinas conectadas atrav�s de uma rede
local, diminuindo assim o tempo e os custos necess�rios para o mesmo.
